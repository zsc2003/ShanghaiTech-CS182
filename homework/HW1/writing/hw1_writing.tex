%!TEX program = xelatex
\documentclass[10pt]{article}
\usepackage{amssymb}
\usepackage{amsmath}
\usepackage{mathrsfs}
\usepackage{titlesec}
\usepackage{xcolor}
\usepackage{enumerate}
\usepackage{bm}
\usepackage{tikz}
\usepackage{listings}
\usetikzlibrary{arrows}
\usepackage{subfigure}
\usepackage{graphicx,booktabs,multirow}
\usepackage[a4paper]{geometry}
\usepackage{upquote}
\usepackage{float}
\usepackage{pdfpages}

\geometry{verbose,tmargin=2cm,bmargin=2cm,lmargin=2cm,rmargin=2cm}
\geometry{verbose,tmargin=2cm,bmargin=2cm,lmargin=2cm,rmargin=2cm}
\lstset{language=Matlab}
\lstset{breaklines}

\input defs.tex

\newtheorem{proposition}{Proposition}
\newtheorem{remark}{Remark}

\titleformat*{\section}{\centering\LARGE\scshape}
\renewcommand{\thesection}{\Roman{section}}
\lstset{language=Matlab,tabsize=4,frame=shadowbox,basicstyle=\footnotesize,
keywordstyle=\color{blue!90}\bfseries,breaklines=true,commentstyle=\color[RGB]{50,50,50},stringstyle=\ttfamily,numbers=left,numberstyle=\tiny,
  numberstyle={\color[RGB]{192,92,92}\tiny},backgroundcolor=\color[RGB]{245,245,244},inputpath=code}

\begin{document}

\date{\today}
\title{Introduction to Machine Learning, Fall 2023 \\
	Homework 1\\
	\small (Due Thursday, Oct. 26 at 11:59pm (CST))}
\maketitle
\begin{enumerate}[1.]


	\item \defpoints{10} [Math review] Suppose $\{\mathbf{X}_1, \mathbf{X}_2, \cdots, \mathbf{X}_n\}$ form a random sample from a multivariate distribution:
	      \begin{itemize}
		      \item[(a)] Prove that the covariance of $\mathbf{X}_i$ is a semi positive definite matrix. ~\defpoints{3}
		      \item[(b)] Assuming $\mathbf{X}_i\sim \mathcal{N}(\mathbf{\mu},\mathbf{\Sigma})$ which is a multivariate normal distribution, and samples $X_i$, derive the the log-likelihood $\mathit{l}(\mathbf{\mu},\mathbf{\Sigma})$ and MLE of $\mathbf{\mu}$ ~\defpoints{4}
		      \item[(c)] Suppose $\hat{\theta}$ is an unbiased estimator of $\theta$ and $\mathbf{Var}(\hat{\theta})>0$. Prove that $(\hat{\theta})^2$ is not an unbiased estimator of $\theta^2$. ~\defpoints{3}
	      \end{itemize}
		
(a)  


(b) 



(c)



	      \newpage

	\item \defpoints{10} Consider real-valued variables $X$ and $Y$, in which $Y$ is generated conditional on $X$ according to
	$$
	Y = aX + b + \epsilon, \ \text{where} \ \epsilon \sim \mathcal{N}(0, \sigma^2).
	$$
	Here $\epsilon$ is an independent variable, called a noise term, which is drawn from a Gaussian distribution with mean 0,
	and variance $\sigma^2$. This is a single variable linear regression model, where $a$ is the only weight parameter and $b$ denotes the intercept.
	The conditional probability of $Y$ has a distribution $p(Y | X, a, b) \sim \mathcal{N}(aX+b, \sigma^2)$, so it can be written as:
	$$
	p(Y|X, a,b) = \frac{1}{\sqrt{2\pi}\sigma}\exp\left(-\frac{1}{2\sigma^2}(Y - aX -b)^2\right).
	$$
	\begin{itemize}
		\item[(a)] Assume we have a training dataset of $n$ i.i.d. pairs $(x_i, y_i)$, $i = 1, 2, ..., n$, and
		the likelihood function is defined by $L(a,b) = \prod_{i=1}^n p(y_i | x_i, a, b)$. Please write the
		Maximum Likelihood Estimation (MLE) problem for estimating $a$ and $b$.~\defpoints{3}
		\item[(b)] Estimate the optimal solution of $a$ and $b$ by solving the MLE problem in (a).~\defpoints{4}
		\item[(c)] Based on the result in (b), argue that the learned linear model $f(X) = aX + b$,
		always passes through the point $(\bar{x},\bar{y})$,
		where $\bar{x} = \tfrac{1}{n}\sum_{i=1}^{n}x_{i}$ and $\bar{y} = \tfrac{1}{n}\sum_{i=1}^{n}y_{i}$ denote the sample means.~\defpoints{3}
	\end{itemize}
(a)



(b)




(c)







	      \newpage

	\item \defpoints{10} [Regression and Classification]
	      \begin{itemize}
			\item[(a)] When we talk about linear regression, what does `linear' regard to? \defpoints{2}
			\item[(b)] Assume that there are $n$ given training examples $\{(x_1, y_1), (x_2, y_2), \cdots, (x_n, y_n)\}$,
			where each input data point $x_i$ has $m$ real valued features. When $m > n$, the linear regression model
			is equivalent to solving an under-determined system of linear equations $\mathbf{y} = \mathbf{X}\beta$. One popular way to
			estimate $\beta$ is to consider the so-called ridge regression:
			\[\argmin_{\beta} ||\mathbf{y}-\mathbf{X}\mathbf{\beta}||_2^2 + \lambda||\beta||_2^2\]
			for some $\lambda > 0$. This is also known as Tikhonov regularization.
			
			Show that the optimal solution $\beta_*$ to the above optimization problem is given by
			\[\mathbf{\beta}_* = (\mathbf{X}^T\mathbf{X} + \lambda \mathbf{I})^{-1}\mathbf{X}^T\mathbf{y}\]
			Hint: You need to prove that given $\lambda>0$, $\mathbf{X}^T\mathbf{X} + \lambda \mathbf{I}$ is invertible. \defpoints{5}
			\item[(c)] Is the given data set linear separable? If yes, construct a linear hypothesis function to separate the given data set. If no, explain the reason. \defpoints{3}
			
			\begin{table}[h]
				\centering
				\begin{tabular}{c|cccccc}
					Data & (1,3) & (4,4) & (3,-6) & (-2,1) & (-3,5) & (-6,-4) \\ \hline
					Label & +1 & -1 & -1 & +1 & -1 & -1
				\end{tabular}
				\label{tab:my_label}
			\end{table}
	      \end{itemize}
(a)




(b)




(c)
      	  
      	  
      	  \newpage


\end{enumerate}

\end{document}