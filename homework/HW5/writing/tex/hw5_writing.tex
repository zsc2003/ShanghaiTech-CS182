%!TEX program = xelatex
\documentclass[10pt]{article}
\usepackage{amssymb}
\usepackage{amsmath}
\usepackage{mathrsfs}
\usepackage{titlesec}
\usepackage{xcolor}
%\usepackage[shortlabels]{enumitem}
\usepackage{enumerate}
\usepackage{bm}
\usepackage{tikz}
\usepackage{listings}
\usetikzlibrary{arrows}
\usepackage{subfigure}
\usepackage{graphicx,booktabs,multirow}
\usepackage[a4paper]{geometry}
\usepackage{upquote}
\usepackage{float}
\usepackage{pdfpages}
\usepackage{amsthm}
\usepackage{bbm}

\geometry{verbose,tmargin=2cm,bmargin=2cm,lmargin=2cm,rmargin=2cm}
\geometry{verbose,tmargin=2cm,bmargin=2cm,lmargin=2cm,rmargin=2cm}
\lstset{language=Matlab}
\lstset{breaklines}

\input defs.tex

\newtheorem{proposition}{Proposition}
\newtheorem{remark}{Remark}

\titleformat*{\section}{\centering\LARGE\scshape}
\renewcommand{\thesection}{\Roman{section}}
\lstset{language=Matlab,tabsize=4,frame=shadowbox,basicstyle=\footnotesize,
keywordstyle=\color{blue!90}\bfseries,breaklines=true,commentstyle=\color[RGB]{50,50,50},stringstyle=\ttfamily,numbers=left,numberstyle=\tiny,
  numberstyle={\color[RGB]{192,92,92}\tiny},backgroundcolor=\color[RGB]{245,245,244},inputpath=code}

\begin{document}

\date{}
\title{Optimization and Machine Learning, Fall 2023 \\
	Homework 5 \\
	\small (Due Thursday, Jan 11 at 11:59pm (CST))}
\maketitle
\begin{enumerate}[1.]

	\item \defpoints{10} [Deep Learning Model]
	\begin{enumerate}
		\item Consider a 2D convolution layer. Suppose the input size is 4 $\times$ 64 $\times$ 64 $\times$ (channel, width, height) and
		we use \textbf{ten} 3 $\times$ 3 (width, height) kernels with 4 channels input and 4 channels output to convolve with it. Set stride = 1 and pad = 1. What is 
		the output size? Let the bias for each kernel be a scalar, how many parameters do we have in this layer? \defpoints{5}
		\item The convolution layer is followed by a max pooling layer with 2 × 2 (width, height) filter and stride
		= 2. What is the output size of the pooling layer? How many parameters do we have in the pooling
		layer? \defpoints{5}
	\end{enumerate}
	
(a)





(b)


	
	
	
	
	
	
	\newpage
	
	\item \defpoints{10} Use the $k$-means++ algorithm and Euclidean distance to cluster the 8 data points into $K=3$ clusters.
	      The coordinates of the data points are:
	      \begin{align*}
		      x^{(1)} & = (2,8),  \ x^{(2)} = (2,5), \ x^{(3)} = (1,2), \ x^{(4)} = (5,8), \\
		      x^{(5)} & = (7,3),  \ x^{(6)} = (6,4), \ x^{(7)} = (8,4), \ x^{(8)} = (4,7).
	      \end{align*}
	      Suppose that initially the first cluster centers is $x^{(1)}$. \\
	      {\color{blue} To ensure consistent results, please use random numbers in the order shown in the table below. When selecting a center, arrange it in ascending order of sequence number. For example, when the normalized weights of 5 nodes are 0.2, 0.1, 0.3, 0.3, and 0.1, if the random number is 0.3, the selected node is the third one. Note that you don't necessarily need to use all of them.\\
	      \begin{tabular}{|c|c|c|c|c|}
	      	\hline
	      	0.6 & 0.2 & 0.5 & 0.9 & 0.3 \\
	      	\hline
	      \end{tabular}
	      }
	      \begin{itemize}
		      \item[(a)] Perform the $k$-means++ algorithm to initialize other centers and report the coordinates of the resulting centroids. ~\defpoints{3}
		      \item[(b)] Calculate the loss function
		            \begin{equation}
			            Q(r,c) = \frac{1}{n} \sum_{i=1}^n \sum_{j=1}^K r_{ij}||x^{(i)} - c_j||^2,
		            \end{equation}
		            where $r_{ij} = 1$ if $x^{(i)}$ belongs to the $j$-th cluster and 0 otherwise. ~\defpoints{2}
		      \item[(c)] How many more iterations are needed to converge? ~\defpoints{3} Calculate the loss after it converged.~\defpoints{2}
	      \end{itemize}
		
(a)



(b)



(c)



		 
		 
		 
		\newpage


	\item \defpoints{10} Name 2 deep generation networks.~\defpoints{2} Briefly describe the training procedure of a GAN model.(What's the objective function? How to update the parameters in each stage?)~\defpoints{8}\\

	
	



\end{enumerate}
\end{document}