%!TEX program = xelatex
\documentclass[10pt]{article}
\usepackage{amsthm}
\usepackage{amssymb}
\usepackage{amsmath}
\usepackage{mathrsfs}
\usepackage{titlesec}
\usepackage{xcolor}
\usepackage{enumerate}
\usepackage{bm}
\usepackage{tikz}
\usepackage{listings}
\usetikzlibrary{arrows}
\usepackage{subfigure}
\usepackage{graphicx,booktabs,multirow}
\usepackage[a4paper]{geometry}
\usepackage{upquote}
\usepackage{float}
\usepackage{pdfpages}
\usepackage{amsthm}
\usepackage{bbm}

\geometry{verbose,tmargin=2cm,bmargin=2cm,lmargin=2cm,rmargin=2cm}
\geometry{verbose,tmargin=2cm,bmargin=2cm,lmargin=2cm,rmargin=2cm}
\lstset{language=Matlab}
\lstset{breaklines}

\input defs.tex

\newtheorem{proposition}{Proposition}
\newtheorem{remark}{Remark}

\titleformat*{\section}{\centering\LARGE\scshape}
\renewcommand{\thesection}{\Roman{section}}
\lstset{language=Matlab,tabsize=4,frame=shadowbox,basicstyle=\footnotesize,
keywordstyle=\color{blue!90}\bfseries,breaklines=true,commentstyle=\color[RGB]{50,50,50},stringstyle=\ttfamily,numbers=left,numberstyle=\tiny,
  numberstyle={\color[RGB]{192,92,92}\tiny},backgroundcolor=\color[RGB]{245,245,244},inputpath=code}

\begin{document}

\date{\today}
\title{Introduction to Machine Learning, Fall 2023 \\
	Homework 3\\
	\small (Due Tuesday Nov. 30 at 11:59pm (CST))}
\maketitle

\begin{enumerate}[1.]

	\item \defpoints{15} [Expectation Maximization Algorithm]
            Consider a probabilistic model in which we collectively denote the observed variables by $\boldsymbol{X}$ and all of the hidden variables by $\boldsymbol{Z}$. The joint distribution $p(\boldsymbol{X},\boldsymbol{Z}|\theta)$ is parameterized by $\theta$. Our goal is to maximize the likelihood function given by
            \begin{equation}
                p(\boldsymbol{X}|\theta).
            \end{equation}
 
	      \begin{itemize}
		      \item[(a)] Given an arbitrary distribution $q$, show that the log-likelihood of $\boldsymbol{X}$ is~\defpoints{5}
                    \begin{equation}
                        \log p(\boldsymbol{X}|\theta) = \mathbb{E}_{\boldsymbol{Z}\sim q}\left [ \log  \frac{p(\boldsymbol{X}, \boldsymbol{Z}|\theta)}{q(\boldsymbol{Z})}\right ] + KL\left(q(\boldsymbol{Z})\| p(\boldsymbol{Z}|\boldsymbol{X},\theta)\right).
                    \end{equation}
		      \item[(b)] Next let's consider the expectation step. First show the evidence lower bound (ELBO) is a lower bound of the log-likelihood, namely~\defpoints{5}
                    \begin{equation}
                        \log p(\boldsymbol{X}|\theta)\geq\mathbb{E}_{\boldsymbol{Z}|\boldsymbol{X},\theta^{(t-1)}}\left[\log \frac{p(\boldsymbol{X},\boldsymbol{Z}|\theta)}{p(\boldsymbol{Z}|\boldsymbol{X},\theta^{(t-1)})}\right],
                    \end{equation}
                    where $\theta^{(t-1)}$ is the parameter estimated in the previous iteration.
		      \item[(c)] We want to maximize the ELBO, $\mathbb{E}_{\boldsymbol{Z}|\boldsymbol{X},\theta^{(t-1)}}\left[\log \frac{p(\boldsymbol{X},\boldsymbol{Z}|\theta)}{p(\boldsymbol{Z}|\boldsymbol{X},\theta^{(t-1)})}\right]$ since maximizing $p(\boldsymbol{X}|\theta)$ is hard. EM algorithm defines $Q(\theta|\theta^{(t-1)}) := \mathbb{E}_{\boldsymbol{Z}|\boldsymbol{X},\theta^{(t-1)}}\left[ \log p(\boldsymbol{X},\boldsymbol{Z}|\theta) \right]$. The M-step is given by:
                    \begin{equation}
                        \theta^{(t)} \leftarrow \text{arg}\max_{\theta} Q(\theta|\theta^{(t-1)}). 
                    \end{equation}
                    Show that maximizing $Q(\theta|\theta^{(t-1)})$ and maximizing the ELBO is equivalent.~\defpoints{5} Formally,
                    \begin{equation}
                        \text{arg}\max_{\theta} Q(\theta|\theta^{(t-1)}) = \text{arg}\max_{\theta} \mathbb{E}_{\boldsymbol{Z}|\boldsymbol{X},\theta^{(t-1)}}\left[\log \frac{p(\boldsymbol{X},\boldsymbol{Z}|\theta)}{p(\boldsymbol{Z}|\boldsymbol{X},\theta^{(t-1)})}\right]
                    \end{equation}
			  % \item[(d)] Prove that the likelihood is guaranteed to increase at each iteration, ~\defpoints{3}
     %                \begin{equation}
     %                    p(\boldsymbol{X}|\theta^{(t)}) \geq p(\boldsymbol{X}|\theta^{(t-1)}).
     %                \end{equation}

     
	      \end{itemize}

		  \textbf{Solution:}

(a) 



(b)





(c)




(d)





(e)












	      \newpage

    \item ~\defpoints{15} [Boosting]
        Suppose that we are interested in learning a classifier, such that at any turn of a game we can pose a question, like ``should I attack this ant hill now?", and get an answer.That is, we want to build a classifier which we can feed some features on the current game state, and get the output ``attack" or ``don't attack". There are many possible ways to define what the action ``attack" means, but for now let's define it as sending all friendly ants that can see the ant hill under consideration towards it.

	    Let's recall the AdaBoost algorithm described in class. Its input is a dataset $\{(x_{i},y_{i})\}_{i=1}^{n}$, with $x_i$ being the $i$-th sample, and $y_{i}\in \{-1,1\}$ denoting the $i$-th label, $i=1,2,...,n$. The features might be composed of a count of the number of friendly ants that can see the ant hill under consideration, and a count of the number of enemy ants these friendly ants can see. For example, if there were 10 friendly ants that could see a particular ant hill, and 5 enemy ants that the friendly ants could see, we would have:
	    \begin{align*}
		    x_1 = \begin{bmatrix}
			10 \\
			5
		    \end{bmatrix}.
	    \end{align*}

	    The label of the example $x_{1}$ is $y_{1} = 1$, once the friendly ants were successful in razing the enemy ant hill, and $y_{1} = 0$ otherwise. We could generate such examples by running a greedy bot (or any other opponent bot) against a bot that we periodically try to attack an enemy ant hill. Each time this bot tries the attack, we record (say, after $20$ turns or some other significant amount of time) whether the attack was successful or not.

	    \begin{itemize}
            \item[(a)] Let $\epsilon_t$ denote the error of a weak classifier $h_t$:
		        \begin{equation}
		            \epsilon_{t} = \sum_{i=1}^{n} D_{t}(i) \mathbbm{1}(y_{i} \neq h_{t}(x_{i})).
		        \end{equation}
		        In the simple “attack” / “don't attack” scenario, suppose that we have implemented the following six weak classifiers:
		        \begin{align*}
		            h^{(1)}(x_{i}) = 2 * \mathbbm{1}(x_{i1} \geq 2) - 1, \hspace{1cm}  & h^{(4)}(x_{i}) = 2 * \mathbbm{1}(x_{i2} \leq 2) - 1,  \\
			        h^{(2)}(x_{i}) = 2 * \mathbbm{1}(x_{i1} \geq 6) - 1, \hspace{1cm}  & h^{(5)}(x_{i}) = 2 * \mathbbm{1}(x_{i2} \leq 6) - 1,  \\
			        h^{(3)}(x_{i}) = 2 * \mathbbm{1}(x_{i1} \geq 10) - 1, \hspace{1cm} & h^{(6)}(x_{i}) = 2 * \mathbbm{1}(x_{i2} \leq 10) - 1. \\
		        \end{align*}
		        Given ten training data points ($n = 10$) as shown in Table 1,
		        \begin{table}[t]
                    \caption{The training data in (a).}
                    \label{table1}
                    \centering
                    \begin{tabular}{|c|cc|c|}
                    \hline
                    $i$ & $x_{i1}$ & $x_{i2}$ & $y_{i}$ \\ \hline
                    1 & 1.5 & 0.5 & 1 \\
                    2 & 2.5 & 1.5 & 1 \\
                    3 & 3.5 & 3.5 & 1 \\
                    4 & 6.5 & 5.5 & 1 \\
                    5 & 7.5 & 10.5 & 1 \\
                    6 & 1.5 & 2.5 & -1 \\
                    7 & 3.5 & 1.5 & -1 \\
                    8 & 5.5 & 5.5 & -1 \\
                    9 & 7.5 & 8.5 & -1 \\
                    10 & 1.5 & 10.5 & -1 \\
                    \hline
                    \end{tabular}
                \end{table}
		        please show that what is the minimum value of $\epsilon_{1}$ and which of $h^{(1)},...,h^{(6)}$ achieve this value? Note that there may be multiple classifiers that all have the same $\epsilon_{1}$. You should list all classifiers that achieve the minimum $\epsilon_{1}$ value.~\defpoints{3}\\

	        \item[(b)] For all the questions in the remainder of this section, let $h_{1}$ denote $h^{(1)}$ chosen in the first round of boosting. (That is, $h^{(1)}$ was the classifier that achieved the minimum $\epsilon_{1}$.)
		        \begin{enumerate}
			        \item[(1)] What is the value of $\alpha_{1}$ (the weight of this first classifier $h_{1}$)? ~\defpoints{1}\\

			        \item[(2)] What should $Z_{t}$ be in order to make sure the distribution $D_{t+1}$ is normalized correctly? That is, derive the formula of $Z_{t}$ in terms of $\epsilon_{t}$ that will ensure $\sum_{i=1}^{n} D_{t+1}(i) = 1$. Please aslo derive the formula of $\alpha_{t}$ in terms of $\epsilon_{t}$. ~\defpoints{3}\\

			        \item[(3)] Which points will increase in significance in the second round of boosting? That is, for which points will we have $D_{1}(i) < D_{2}(i)$? What are the values of $D_{2}$ for these points?~\defpoints{3}\\

			        \item[(4)] In the second round of boosting, the weights on the points will be different, and thus the error $\epsilon_2$ will also be different. Which of $h^{(1)}, . . . , h^{(6)}$ will minimize $\epsilon_2$? (Which classifier will be selected as the second weak classifier $h_2$?) What is its value of $\epsilon_2$?~\defpoints{3}\\

			        \item[(5)] What will the average error of the final classifier $H$ be, if we stop after these two rounds of boosting? That is, if $H(x) = \text{sign}(\alpha_{1}h_{1}(x) + \alpha_{2}h_{2}(x))$, what will the  training error $\epsilon = \frac{1}{n} \sum_{i=1}^{n} \mathbbm{1} (y_{i} \neq h(x_{i}))$ be? Is this more, less, or the same as the error we would get, if we just used one of the weak classifiers instead of this final classifier $H$?~\defpoints{2}\\

		        \end{enumerate}
        \end{itemize}
		\textbf{Solution:}

(a)




(b)\\
(1)




(2)




(3)




(4)




(5)









        \newpage

	\item \defpoints{10} [Perceptron Learning Algorithm]
            Consider a binary classification problem. The input space is $\mathbb{R}^{d}$. The output space is $\{ +1, -1 \}$. For simplicity, we modified the input to be $\mathbf{x} = [x_0, x_1, \cdots, x_d]^{\intercal}$ with $x_0=1$. The output is predicted using the hypothesis:
            \begin{equation}
                h(\mathbf{x}) = \text{sign}(\mathbf{w}^{\intercal}\mathbf{x}),
            \end{equation}
            where $\mathbf{w} = [w_0, w_1, \cdots, w_d]^{\intercal}$ and $w_0$ is the bias.
            
            The \textit{perceptron learning algorithm} determines $\mathbf{w}$ using a simple iterative method. Here is how it works. At iteration $t$, where $t=0,1,2, \ldots$, there is a current value of the weight vector, call it $\mathbf{w}(t)$. The algorithm picks an example from $\left(\mathbf{x}_1, y_1\right) \cdots\left(\mathbf{x}_N, y_N\right)$ that is currently misclassified, call it $(\mathbf{x}(t), y(t))$, and uses it to update $\mathbf{w}(t)$. Since the example is misclassified, we have $y(t) \neq$ $\operatorname{sign}\left(\mathbf{w}^{\mathrm{T}}(t) \mathbf{x}(t)\right)$. The update rule is

            \begin{equation}
                \mathbf{w}(t+1)=\mathbf{w}(t)+y(t) \mathbf{x}(t).    
            \end{equation}

                
		\begin{itemize}
			\item[(a)] Show that $y(t) \mathbf{w}^{\mathrm{T}}(t) \mathbf{x}(t)<0$. [Hint: $\mathbf{x}(t)$ is misclassified by $\mathbf{w}(t)$.]~\defpoints{3} 
			\item[(b)] Show that $y(t) \mathbf{w}^{\mathrm{T}}(t+1) \mathbf{x}(t)>y(t) \mathbf{w}^{\mathrm{T}}(t) \mathbf{x}(t)$. [Hint: Use (1.3).]~\defpoints{3} 
			\item[(c)]   As far as classifying $\mathbf{x}(t)$ is concerned, argue that the move from $\mathbf{w}(t)$ to $\mathrm{w}(t+1)$ is a move ``in the right direction".~\defpoints{4} 
		\end{itemize}

		\textbf{Solution:}
        
(a)




(b)



(c)




\end{enumerate}

\end{document}